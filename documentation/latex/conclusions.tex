%-------------------------------------------------------------------------------
% File: conclusions.tex
%       Part of StockSim project documentation.
%       See main.tex for further information.
%-------------------------------------------------------------------------------
\chapter{Conclusions}
The modular architecture of the project (Chapter 5) allowed us to create
stand-alone APIs (for both the databases and Yahoo Finance) that are not
inherently tied with the StockSim application or with financial applications in
general. They all can possibly be used stand-alone and could be integrated in
different projects.\\
\\
Most of the choices we made during both the designing and the development
process proved to be effective. We opted for a column database architecture to
store the historical data because it really lends itself to such usages.
Performance analysis of the query latency (as shown in paragraph 4.5) confirmed
our hypothesis, although tests have been run only locally.\\
It would be interesting to repeat the measurements with a bigger database and in
a controlled environment where network jittering does not skew the results too
much, something that we were not able to do due to limited resources of the VMs
and due to the fact that all traffic is filtered through a VPN.\\
\\
At the moment the database is limited to the US market data only, but it would
be really simple to integrate data from other stock markets around the world
using the StockSim Client in \texttt{admin} mode.\\
From the Software and Data Model architectural point of view, this would not be
a problem since the column database architecture is by nature suitable for
geographical distribution and for dealing with huge amounts of data. Another
test worth doing to evaluate the performance of the application would be to
deploy several geographically distributed servers and increase the data volume
by some orders of magnitude: the current database size is around 1.43 GB and it
would be interesting to see how it behaves with tens of terabytes of data.
