%-------------------------------------------------------------------------------
% File: requirements.tex
%       Part of StockSim project documentation.
%       See main.tex for further information.
%-------------------------------------------------------------------------------
\chapter{Actors and requirements}
The main actors and the functional requirements can be defined using the simple
application description provided in the introductory chapter. Non-functional
requirements, described in detail in the second section of this chapter, are the
characteristics that ensure satisfactory interaction with users and real world
utility of the product.
\section{Actors}
Based on the application design, four main actors can be identified:
\begin{itemize}
    \item A \textbf{Guest user} is someone who is not yet registered (does not
    have a valid account to be able to use the application); this actor does not
    own private credentials for the login; in order to exploit the application
    main functionalities, he must register a new user account; therefore, the
    registration is the only action allowed for a Guest;
    \item A \textbf{Registered user} is someone who signed up on the application;
    this actor owns private credentials (username and password) for the login;
    he can login as a user into the client application and utilize its main
    features to search for stocks, get historical information about them,
    compose portfolios and simulate them.
    \item An \textbf{Admin user} is someone who is registered on the application
    with administration credentials; this actor can login as an admin using the
    StockSim client running in \texttt{admin} mode to perform maintenance
    operations; these operations include, among others, adding a new stock to
    the database, adding new admin credentials, removing normal and admin users;
    \item The \textbf{Data Updater} is a thread running on StockSim Server; this 
    thread is supposed to be always running, and it is in charge of updating the
    database historical data with the new information coming from the stock
    market daily trading sessions; it is also in charge of finding and fixing
    (or at least report) data integrity issues.
\end{itemize}
Please refer to Chapter 3 for the full use case diagram.

\section{Requirements}
\subsection{Functional requirements}
\begin{itemize}
    \item The application is available to be used only by registered users;
    \item The application provides access to stocks and ETFs historical data, day by day, starting from 2010;\\
	\item A guest user, upon launching the application, should be able to sign-up;
	\item A registered user should be able to and sign-in;
	\item A registered user should be able to search for a stock and view related details;
	\item A registered user should be able to view charts of the historical data of a stock;
	\item A registered user should be able to create and delete portfolios;
	\item A registered user should be able to run simulations on a portfolio;
	\item A registered user should be able to visualize statistics about a simulation and view related charts.\\
	\item Only admins can add new stocks to the database;
	\item Only admins can create new admin accounts;
	\item Only admins can delete admin accounts;
	\item Only admins can delete user accounts.
\end{itemize}
\subsection{Non-functional requirements}
\begin{itemize}
    \item The terminal based menu should allow the user to accomplish the desired tasks with the minimum amount of commands;
    \item The application will store information in non-relation databases (MongoDB and Apache Cassandra);
    \item Stocks historical data and information should be always available and updated to the latest market trading session;
    \item MongoDB and Apache Cassandra stocks data should always be consistent;
	\item The retrieval of stocks and portfolios historical data and information should be fast;
	\item Charting retrieved data should be fast;
	\item User password should be stored in a secure way (i.e. hashed).
\end{itemize}
