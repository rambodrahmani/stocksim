%-------------------------------------------------------------------------------
% File: requirements.tex
%       Part of StockSim project documentation.
%       See main.tex for further information.
%-------------------------------------------------------------------------------
\chapter{Actors and requirements}
The main actors and the functional requirements are defided based on the previous
simple description of the application. There is also a 
particular actor witch is in charge of the authomatic update of the dataset; non functional
requirements are the caracteristics that ensure satisfactory interaction with users and 
real world utility of the product.

\section{Actors}
Based on the application design, four actors can interact with the system:
\begin{itemize}
    \item A \textbf{Guest user} is someone who's not registered on the applicazion; this actor doesn't 
own private credentials for the login; in order to exploit the application main functionalities, 
it has to register a new account. The registration it's the only action allowed for a Guest;

    \item A \textbf{Registered user} is someone who's registered on the applicazion; this actor own
private credentials (username and password) for the login; this actor can login as a user
into the client
application and utilize it's main features like watch stoks information, compose portfolios and 
simulate them. 
   
    \item An \textbf{Admin user} is someone who's registered on the applicazion with administration
credentials; this actor can login as an admin into the client and do some maintenance 
operations; this operations includes check the integrity of the entire dataset and add new 
stocks to it;

    \item The \textbf{Data Updater} is a thread running on one server; this thread is suppose to 
run always, and it's in charge of update the dataset with the new information coming from 
the stock market daily sessions; it's also in charge to find and fix (at list report)
integrity issues.

\end{itemize}

The full use case diagram is provided on chapter 3.
\section{Requirements}

\subsection{Functional requirements}
Something about functional requirements.

\subsection{Non-functional requirements}
Something about non-functional requirements.
