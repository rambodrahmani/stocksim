%-------------------------------------------------------------------------------
% File: software_architecture.tex
%       Part of StockSim project documentation.
%       See main.tex for further information.
%-------------------------------------------------------------------------------
\chapter{Software architecture}

In this section we show the general software architecture of the application.\\
Stocksim is a multi-module application written in Java. Its development was helped by the IntelliJ IDEA\footnote{https://www.jetbrains.com/idea/} IDE, together with the build automation tool Maven\footnote{https://maven.apache.org/}.\\
We opted for a multi-module approach so that logically independent sections could be kept separated and better maintained. Furthermore, we decided to follow a feature-oriented development process and the modularity of the project helped us to coding with more ease and efficiency.\\
The entire codebase can be found at https://github.com/rambodrahmani/stocksim .

\section{Maven Multi-Module Structure}

The application consists of three modules: Client, Server and Library.\\
Every module has its own .pom file, necessary for the building process and for keeping track of all the dependencies needed.\\
There is also a parent .pom that belongs to the 'parent' module Stocksim (which encapsulates the whole application), which contains all the dependencies and build settings that are common for all the modules.\\
All .pom files are available in Appendix A.\\
For every module an independent .jar file can be generated  which allows to run that module \textit{standalone}, without any dependency issue.

\subsection{Client}
The Client module is the main core of the application. It is that part of the application which is thought to be distributed to end users.\\
It has the following structure:
\pagebreak
\dirtree{%
.1 CLIENT.
.2 src.
.3 main.
.4 java.
.5 it.unipi.lsmsdb.stocksim.client.
.6 admin.
.6 app.
.6 charting.
.6 database.
.6 user.
}
\hfill \break
\noindent The \texttt{app} directory contains the class with the entry point for the applications.\\
Since the Client program was thought to be possibly launched in two different modes (namely \texttt{admin} or \texttt{user} mode), these two functionalities were split into the homonym directories.\\
The \texttt{database} directory contains the classes necessary for the interaction with the databases and those necessary to store data retrieved from the formers.\\
The \texttt{charting} directory contains an API for the JFreeChart library, used by the Client to create charts of various types.

\subsection{Server}
The Server module takes care of keeping the data updated.\\
It is not supposed to be distributed: it should be always-on so that every day it can pull down from  NasdaqTrader the most recent values about every stocks.
The Server module has the following structure:
\dirtree{%
.1 SERVER.
.2 src.
.3 main.
.4 java.
.5 it.unipi.lsmsdb.stocksim.server.
.6 app.
.6 database.
}
\hfill \break
\noindent The \texttt{app} directory contains the class with the entry point for the application.\\
The \texttt{database} directory contains the classes necessary for the interaction with the databases.

\subsection{Library}
There is also a third module, the \texttt{Library} module, which contains the database APIs both for MongoDB and Cassandra, together with the YahooFinance API and some utility functions.
It has the following structure:
\pagebreak
\dirtree{%
.1 LIBRARY.
.2 src.
.3 main.
.4 java.
.5 it.unipi.lsmsdb.stocksim.lib.
.6 database.
.7 cassandra.
.7 mongoDB.
.6 util.
.6 yfinance.
}
\hfill \break
The \texttt{database} directory contains the two APIs we wrote for the interaction with Cassandra and MongoDB.\\
The \texttt{yfinance} directory contains the API we used to retrieve data from Yahoo Finance.\\
The \texttt{util} directory contains some utilities functions, such as the ones used to set the log level both for Cassandra and Mongo, and an argument parser.

\section{Apache Maven Assembly Plugin}
The Assembly Plugin for Maven enables developers to combine project output into 
a single distributable archive that also contains dependencies, modules, site 
documentation, and other files.
