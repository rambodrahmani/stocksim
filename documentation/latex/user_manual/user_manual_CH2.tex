%-------------------------------------------------------------------------------
% File: user_manual_CH2.tex
%       Part of StockSim project documentation.
%       See main.tex for further information.
%-------------------------------------------------------------------------------
\chapter{StockSim Client Manual}
The StockSim Client has two different running modes: the first one
\vspace{0.2cm}
\begin{lstlisting}[basicstyle=\footnotesize\ttfamily,language={},numbers=left,keepspaces=true,tabsize=4,
numberstyle=\footnotesize,numbersep=8pt,frame=single]
Welcome to the StockSim Client.

*** [RUNNING IN USER MODE] ***

Available Commands:
register		create a new user account.              
login			login to your user account.             
quit			quit StockSim client. 
> 
\end{lstlisting}
\vspace{-0.5cm}
is the \texttt{user mode}. Whereas, the second running mode can be triggered 
using the \texttt{--admin} command line argument:
\vspace{0.2cm}
\begin{lstlisting}[basicstyle=\footnotesize\ttfamily,language={},numbers=left,keepspaces=true,tabsize=4,
numberstyle=\footnotesize,numbersep=8pt,frame=single]
$ java -jar Client.jar --admin

Welcome to the StockSim Client.

*** [RUNNING IN ADMIN MODE] ***

Available Commands:
login            login to your admin account.            
quit             quit StockSim client.                   
> 
\end{lstlisting}
\vspace{-0.5cm}
and is the \texttt{admin mode}. Although they might look like the same, the 
available commands, once the user/admin login has been executed, differ.
\section{StockSim Client Admin Mode}
After launching the StockSim Client in \textit{admin} mode and logging in with
admin credentials (\textit{username}: \texttt{admin}, \textit{password}:
\texttt{stocksim}), the following commands are available:
\begin{lstlisting}[basicstyle=\footnotesize\ttfamily,language={},numbers=left,keepspaces=true,tabsize=4,
numberstyle=\footnotesize,numbersep=8pt,frame=single]
> login
Username: admin
Password: stocksim
Admin login executed correctly.
Welcome StockSim Admin.

Available Commands:
add-stock          add a new stock to the database using its ticker.
add-admin          create new admin account.               
remove-admin       delete admin account.                   
remove-user        delete user account.                    
logout             logout from current admin account.      
quit               quit StockSim client.                   
> 
\end{lstlisting}
\vspace{-0.5cm}
\subsection{Add stock}
The \texttt{add-stock} command allows an admin to add a new stock to the
database, using its ticker (stock symbol) provided that it exists in the
Yahoo! Finance database.\\
The underlying function retrieves both general and historical data from the
YFinance databases and loads it into the application databases (MongoDB and
Apache Cassandra).
\begin{lstlisting}[basicstyle=\footnotesize\ttfamily,,language={},numbers=left,keepspaces=true,tabsize=4,
numberstyle=\footnotesize,numbersep=8pt,frame=single]
> add-stock
Ticker symbol: LSMSDB
Yahoo Finance Quotes not found. Some information will be missing.
Yahoo Finance Asset Profile not found. Some information will be missing.
Could not add new ticker.

> add-stock
Ticker: RACE
Stock already available in the database.
Could not add new stock.

> add-stock
Ticker: G.MI
Asset Profile created with success. Updating historical data.
Historical data updated with success.
New stock added.
\end{lstlisting}
\vspace{-0.5cm}
\subsection{Add admin}
The \texttt{add-admin} option allows an admin to add other admin.
\begin{lstlisting}[basicstyle=\footnotesize\ttfamily,language={},numbers=left,keepspaces=true,tabsize=4,
numberstyle=\footnotesize,numbersep=8pt,frame=single]
> add-admin
Admin account name: John
Admin account surname: Doe
Admin account username: admin2
Admin account password: stocksim
New admin account created.
\end{lstlisting}

\subsection{Remove admin}

The \texttt{remove-admin} option allows an admin to remove another admin from their role.

\begin{lstlisting}[basicstyle=\footnotesize\ttfamily,language={},numbers=left,keepspaces=true,tabsize=4,
numberstyle=\footnotesize,numbersep=8pt,frame=single]
> remove-admin
Admin account username: admin2
Admin account password: stocksim
Admin account deleted.
\end{lstlisting}

\subsection{Remove user}
The \texttt{remove-user} option allows an admin to delete a user from the database.

\begin{lstlisting}[basicstyle=\footnotesize\ttfamily,language={},numbers=left,keepspaces=true,tabsize=4,
numberstyle=\footnotesize,numbersep=8pt,frame=single]
> remove-user
User account email: jsmith@example.com
User account deleted.
\end{lstlisting}

\section{StockSim Client User Mode}
Upon launching the StockSim Client in \textit{user} mode, the user is presented
with three options: \textit{register}, \textit{login} and \textit{quit}.\\
Using the \textit{register} command, the user can sign-up:
\begin{lstlisting}[basicstyle=\footnotesize\ttfamily,language={},numbers=left,keepspaces=true,tabsize=4,
numberstyle=\footnotesize,numbersep=8pt,frame=single]
> register
Name: John
Surname: Smith
E-Mail: jsmith@example.com
Username [login]: jsmith
Password [login]: hunter2
User sign up executed correctly. You can now login.
\end{lstlisting}
\vspace{-0.5cm}
Once the user is registered and logged in, the application offers several commands:
\vspace{0.2cm}
\begin{lstlisting}[basicstyle=\footnotesize\ttfamily,language={},numbers=left,keepspaces=true,tabsize=4,
numberstyle=\footnotesize,numbersep=8pt,frame=single]
> login
Username: jsmith
Password: hunter2
User login executed correctly.
Welcome John Smith.

[jsmith] Available Commands:
search-stock			search for a stock ticker.              
view-stock				view historical data for a stock ticker.
create-portfolio		create a new stock portfolio.           
list-portfolios			list user stock portfolios.             
view-portfolio			view user stock portfolio info.         
simulate-portfolio		simulate user stock portfolio.          
delete-portfolio		delete user stock portfolio.            
logout					logout from current user account.       
quit					quit StockSim client.   
\end{lstlisting}

\subsection{Search stock}
The \texttt{search-stock} option allows the user to search for a specific stock in the database. The search can be done by \textbf{symbol}, by \textbf{sector} or by \textbf{country}.\\

\begin{lstlisting}[basicstyle=\footnotesize\ttfamily,language={},numbers=left,keepspaces=true,tabsize=4,
numberstyle=\footnotesize,numbersep=8pt,frame=single]
> search-stock
[jsmith] Available Search Commands:
symbol-search			search for a stock ticker using its ticker.
sector-search			search for a stock ticker using the sector.
country-search			search for a stock ticker using the country.
\end{lstlisting}
A search by symbol prompts the user for the symbol of the stock to be searched and returns all the information available in the database about that stock.
\begin{lstlisting}[basicstyle=\footnotesize\ttfamily,language={},numbers=left,keepspaces=true,tabsize=4,
numberstyle=\footnotesize,numbersep=8pt,frame=single]
> symbol-search
Ticker Symbol: TSLA
Short Name: Tesla, Inc.
Long Name: Tesla, Inc.
Symbol: TSLA
Quote type: EQUITY
Market capitalization: 8.00041992192E11
PE ratio: 1265.3346
Market: us_market
Exchange timezone short name: EST
Exchange timezone name: America/New_York
Sector: Consumer Cyclical
Industry: Auto Manufacturers
Currency: USD
Location:  3500 Deer Creek Road Palo Alto CA United States
650-681-5000
Logo URL: https://logo.clearbit.com/tesla.com
Website: http://www.tesla.com
Long business summary:
Tesla, Inc. designs, develops, manufactures, leases, and sells electric
vehicles, and energy [...]

\end{lstlisting}

A search by sector opens a two bar charts showing aggregated data for all sectors (one for total market capitalizaion and the other for average trailing P/E) and prompts the user for the sector they are interested in.
Once the desired sector is entered, a list with all related symbols is returned.

\hfill \break
{\centering
\includegraphics[scale=0.28]{img/user_manual/sector_aggregation.png}\\
}

\begin{lstlisting}[basicstyle=\footnotesize\ttfamily,language={},numbers=left,keepspaces=true,tabsize=4,
numberstyle=\footnotesize,numbersep=8pt,frame=single]
> sector-search
Sector Name: Energy
[ BROG, PSXP, KOS, GMLP, CLMT, NCSM, PFIE, CCLP, NGS, WES, ENSV, FTSI, AXAS, EC, DEN, TTI, NBLX, E, GTE, PSX, PED, NNA, VVV, PVL, AR, HP, CEQP, MUR, DK, RTLR, LEU, NGL, NFG, PTEN, MMLP, PAGP, NESR, NR, PBFX, TRMD, BKR, NOG, ... ]
\end{lstlisting}

The search by country is similar to the search by sector: it also opens two bar charts with aggregated data by country and returns a list of the symbols of all the stocks belonging to the specified country.
\begin{lstlisting}[basicstyle=\footnotesize\ttfamily,language={},numbers=left,keepspaces=true,tabsize=4,
numberstyle=\footnotesize,numbersep=8pt,frame=single]
> country-search
Country Name: Italy
[ E, NTZ, KLR, RACE, ]
\end{lstlisting}
\subsection{View stock}

The \texttt{view-stock} option allows the user to see the evolution of a stock in the market for a specific time range.\\
It asks the user for the stock symbol, the start and end dates and the day granularity, and then shows both a candlestick chart and a line chart of that stock for the desired time interval, along with printing all the information about that stock.

\begin{lstlisting}[basicstyle=\footnotesize\ttfamily,language={},numbers=left,keepspaces=true,tabsize=4,
numberstyle=\footnotesize,numbersep=8pt,frame=single]
> view-stock
Ticker Symbol: TSLA
Start Date [YYYY-MM-DD]: 2021-01-01
End Date [YYYY-MM-DD]: 2021-01-10
Days granularity: 1
Short Name: Tesla, Inc.
Long Name: Tesla, Inc.
Symbol: TSLA
Quote type: EQUITY
Market capitalization: 8.00041992192E11
PE ratio: 1265.3346
Market: us_market
Exchange timezone short name: EST
Exchange timezone name: America/New_York
Sector: Consumer Cyclical
Industry: Auto Manufacturers
Currency: USD
Location:  3500 Deer Creek Road Palo Alto CA United States
650-681-5000
Logo URL: https://logo.clearbit.com/tesla.com
Website: http://www.tesla.com
Long business summary:
Tesla, Inc. designs, develops, manufactures, leases, and sells electric
vehicles, and energy [...]
\end{lstlisting}

\hfill \break
{\centering
\includegraphics[scale=0.32]{img/user_manual/view_stock.png}\\
}

\subsection{Create portfolio}
The \texttt{create-portfolio} option allows the user create a new portfolio and insert stocks into it.\\
The user is first required to insert a name for the new portfolio and then to type a list of comma-separated symbols of the stocks they wish to insert in the portfolio.

\begin{lstlisting}[basicstyle=\footnotesize\ttfamily,language={},numbers=left,keepspaces=true,tabsize=4,
numberstyle=\footnotesize,numbersep=8pt,frame=single]
> create-portfolio
Portfolio name: Portfolio1
Ticker Symbols [comma separated]: TSLA, MSFT, RACE
Portfolio created correctly.

\end{lstlisting}

\subsection{List portfolios}
The \texttt{list-portfolios} option shows the user a list of all their portfolios and the stocks each one of them is made of.\\

\begin{lstlisting}[basicstyle=\footnotesize\ttfamily,language={},numbers=left,keepspaces=true,tabsize=4,
numberstyle=\footnotesize,numbersep=8pt,frame=single]
> list-portfolios
Portfolio1: [ TSLA, MSFT, RACE, ]
FAANG: [ FB, AAPL, AMZN, NFLX, GOOGL, ]
\end{lstlisting}

\subsection{View portfolio}
The \texttt{view-portfolio} option allows the user to view the composition of one of their portfolios. It prompts the user for the name of the portfolio to be viewed and then shows a pie chart of the portfolio.\\

{\centering
\includegraphics[scale=0.32]{img/user_manual/view_portfolio.png}\\
}

\subsection{Simulate portfolio}
TODO

\subsection{Delete portfolio}
The \texttt{delete-portfolio} option allows the user to delete one of their portfolios. It prompts the user for the name of the portfolio to be deleted and then it deletes the portfolio.

