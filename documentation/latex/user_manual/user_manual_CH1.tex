%-------------------------------------------------------------------------------
% File: user_manual_CH1.tex
%       Part of StockSim project documentation.
%       See main.tex for further information.
%-------------------------------------------------------------------------------
\chapter{StockSim Server Manual}
For an application dealing with the stock market, it is essential to 
always provide up-to-date, consistent and reliable data. This is the purpose of 
the \textbf{StockSim Server} program. It is not intended to be distributed to 
end users. It is thought to be running 24/7.\\
\\
The StockSim Server has two different startup modes: the first one
\begin{lstlisting}[basicstyle=\footnotesize\ttfamily,language={},numbers=left,
numberstyle=\footnotesize,numbersep=8pt,frame=single]
$ java -jar Server.jar

Welcome to the StockSim Server.

DATA CONSISTENCY CHECK SUCCESS. PROCEEDING WITH UPDATE.

Updating historical data for: IRBT.
Last update date: 2020-12-30.
Days since last update: 2.
Historical data updated for IRBT. Moving on.

...

Updating historical data for: EWU.
Last update date: 2020-12-28.
Days since last update: 4.
Historical data updated for EWU. Moving on.

...


Updating historical data for: XRX.
Last update date: 2020-12-28.
Days since last update: 4.
Historical data updated for XRX. Moving on.

Historical data update terminated.
Elapsed time: 1 hrs, 37 mins, 3622 secs.
Exceptions during update process: 8.
Failed updates: [ARKG, HDS, IWFH, LDEM, WWW, XVV, GINN, AAAU].

Available Commands:

status		check databases status.
update		update databases historical data.
quit		quit Stocksim server.
> 
[UPDATER THREAD] Current New York time: 2021-01-02T07:50-[America/New_York]
[UPDATER THREAD] Going to sleep for 13 hours before next update.
> 
\end{lstlisting}
\vspace{-0.5cm}
which executes the \texttt{historical data update} procedure right after
startup. Whereas, the second startup mode can be triggered using the
\texttt{--no-update} command line argument:
\vspace{0.2cm}
\begin{lstlisting}[basicstyle=\footnotesize\ttfamily,language={},numbers=left,keepspaces=true,tabsize=4,
numberstyle=\footnotesize,numbersep=8pt,frame=single]
$ java -jar Server.jar --no-update

Welcome to the StockSim Server.

Available Commands:
status			check databases status.
update			update databases historical data.
quit			quit Stocksim server.
> 
[UPDATER THREAD] Current New York time: 2021-01-02T06:29-[America/New_York]
[UPDATER THREAD] Going to sleep for 14 hours before next update.
> update
DATA CONSISTENCY CHECK SUCCESS. PROCEEDING WITH UPDATE.

Updating historical data for: AUPH.
Last update date: 2020-12-31.
Days since last update: 1.
Historical data for AUPH already up to date. Moving on.

...

Updating historical data for: EWU.
Last update date: 2020-12-31.
Days since last update: 1.
Historical data for EWU already up to date. Moving on.

...

Updating historical data for: XRX.
Last update date: 2020-12-31.
Days since last update: 1.
Historical data for XRX already up to date. Moving on.

Historical data update terminated.
Elapsed time: 0 hrs, 4 mins, 44 secs.
Exceptions during update process: 3.
Failed updates: [ARKG, XVV, DUAL].

Available Commands:
status       check databases status.
update       update databases historical data.
quit         quit Stocksim server.
> 
\end{lstlisting}
\vspace{-0.5cm}
in this mode, no update is executed right after startup, the main menu is shown
and the user can decide the action to be performed.\\
\\
In the previously shown examples, we should pay attention to the following
\begin{itemize}
    \item when the \texttt{update} command is used, the first output is
    "\texttt{DATA CONSISTENCY CHECK SUCCESS. PROCEEDING WITH UPDATE.}", this is
    because before updating the historical data for each stock an integrity
    check is performed to check the consistency of the data stored on MongoDB
    and Apache Casssandra; the same operation can be executed standalone with
    the command \texttt{status}; in case of inconsistencies, the program will
    show debugging information useful to find out the issues;
    \item the historical data update functionality was implemented to be as
    automated as possible; for each stocks, the StockSim Server automatically
    detects the last update date, the number of days since the last update and
    fetches the required missing data using the Yahoo Finance API;
    \item while the historical data update is being performed, some logs are
    printed relative to the stock being updated, the number of days since the
    last update; also, at the end, a report is provided in order to check for
    eventual errors in the update process; the log level can be increased to be
    more verbose using the \texttt{setLogLevel} utility method provided by
    \texttt{ServerUtil.java};
    \item the \texttt{[UPDATER THREAD]}, which prints messages to the terminal
    in parallel to the \texttt{main} thread, is in charge of executing the
    historical data update each day at 20:00 P.M. (at the end of the
    \textbf{after-hours trading}) New York time.
\end{itemize}
