%-------------------------------------------------------------------------------
% File: introduction.tex
%       Part of StockSim project documentation.
%       See main.tex for further information.
%-------------------------------------------------------------------------------
\chapter{Introduction}
StockSim is a Java application which, as main feature, allows users to simulate 
stock market portfolios. The StockSim application is composed by two main 
programs:
\begin{itemize}
    \item \textbf{StockSim Server}: supposed to be running 24/7 to ensure 
            historical data is always up-to-date;
    \item \textbf{StockSim Client}: can be launched in either \texttt{admin} or 
            \texttt{user} mode.
\end{itemize}
The StockSim Server is not thought to be distributed to end users, whereas the 
StockSim Client can be used by both administrators and normal users. The choice 
was made to provide the same program to both administrators and normal users 
with two different running modes. Administrators can add new ticker symbols, 
new administrator accounts, delete both administrator and normal user accounts. 
Normal users have access to stocks and ETFs historical data, day by day, 
starting from 2010. They can create their own stock portfolios, run simulations 
and visualize the resulting statistics.\\
\\
Before continuing with what follows, the following terms should be clarified:
\begin{itemize}
    \item the \textbf{stock market} is any exchange that allows people to buy 
and sell stocks and companies to issue stocks; a stock represents the company's 
equity, and shares are pieces of the company;
    \item a collection of investments owned by an investor makes up his or her 
\textbf{portfolio}; you can have as few as one stock in a portfolio, but you 
can also own an infinite amount of stocks or other securities;
    \item a \textbf{stock symbol} is a one- to four-character alphabetic root 
symbol that represents a publicly traded company on a stock exchange; Apple's 
stock symbol is AAPL, while Walmart's is WMT;
    \item the NYSE and Nasdaq are open from Monday through Friday 9:30 A.M. to 
4:00 P.M. (eastern time);
    \item the NYSE and Nasdaq close at 4 P.M., with after-hours trading 
continuing until 8 P.M.; the close simply refers to the time at which a stock 
exchange closes to trading;
    \item trading stocks after normal market hours through an electronic market, 
typically between 4:05 and 8:00 P.M., is \textbf{after-hours trading};
    \item the \textbf{high} is the highest price at which a stock traded during 
a period;
    \item the \textbf{low} is the lowest price of the period;
    \item \textbf{open} means the price at which a stock started trading when 
the opening bell rang; it can be the same as where the stock closed the night 
before, but not always; sometimes events such as company earnings reports that 
happen in after-hours trading can alter a stock’s price overnight;
    \item \textbf{close} refers to the price of an individual stock when the 
stock exchange closed shop for the day; it represents the last buy-sell order 
executed between two traders; in many cases, this occurs in the final seconds of 
the trading day;
    \item the \textbf{adjusted closing price} amends a stock's closing price to 
reflect that stock's value after accounting for any corporate actions; a 
stock's price is typically affected by some corporate actions, such as 
stock splits, dividends, and rights offerings; adjustments allow investors to 
obtain an accurate record of the stock's performance;
    \item \textbf{volume} is the total number of shares traded in a security 
over a period; every time buyers and sellers exchange shares, the amount gets 
added to the period’s total volume.
\end{itemize}
